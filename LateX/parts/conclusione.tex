\chapter*{Conclusione}
\markboth{\MakeUppercase{Conclusione}}{\MakeUppercase{Conclusione}}
\addcontentsline{toc}{chapter}{Conclusione}
I Large Language Models (LLM) rappresentano una delle più significative innovazioni tecnologiche nel campo dell'intelligenza artificiale e dell'elaborazione del linguaggio naturale (NLP). Grazie alla loro capacità di apprendere da enormi quantità di dati e di generalizzare le informazioni apprese, questi modelli sono diventati strumenti potenti in una vasta gamma di applicazioni, capaci di generare testi di qualità paragonabile a quelli scritti da esseri umani. Le loro capacità, tuttavia, non si limitano alla generazione di testo: con il supporto di tecniche avanzate come la Retrieval-Augmented Generation (RAG), gli LLM possono essere ulteriormente potenziati per risolvere problemi complessi, gestire conversazioni personalizzate e interagire con fonti di conoscenza esterne.

In questa tesi sono state esplorate e discusse diverse tecniche per migliorare le prestazioni e le funzionalità degli LLM. Strategie come il Fine Tuning, l'Instruction Tuning e il Prompt Tuning sono state confrontate con la RAG, evidenziando come quest'ultima risulti particolarmente efficace quando si tratta di aumentare le conoscenze del modello senza doverlo ri-addestrare completamente. La RAG permette di integrare il processo di retrieval in tempo reale, dando così al LLM la possibilità di accedere a un corpus di conoscenze dinamico e in costante aggiornamento.

Il fulcro di questo documento è stato la progettazione e lo sviluppo di OmniBot, un prototipo di chatbot basato su un'architettura personalizzata di RAG. OmniBot è stato ideato con l'obiettivo di assistere l'utente nella ricerca di informazioni in settori specifici, consentendogli di accedere rapidamente e in modo efficiente a un'ampia gamma di dati pertinenti. Il sistema ha dimostrato un'elevata capacità di generare risposte coerenti, pertinenti e contestualizzate, rivelandosi uno strumento versatile e adattabile a molteplici scenari applicativi.

La sua capacità di personalizzazione e adattamento ai diversi contesti ha confermato l'efficacia dell'approccio RAG per il miglioramento delle prestazioni degli LLM. In particolare, l'integrazione del retrieval dinamico ha consentito al chatbot di accedere e fornire informazioni aggiornate, superando le limitazioni dei modelli statici che devono essere continuamente addestrati con nuovi dati.

Il lavoro presentato ha contribuito a dimostrare come le tecnologie basate su LLM e RAG possano essere applicate in contesti reali per risolvere problemi complessi e migliorare l'esperienza utente. OmniBot non solo rappresenta un avanzamento significativo nel campo dei chatbot, ma pone le basi per ulteriori sviluppi in ambiti come l'assistenza automatizzata, l'e-commerce e la gestione di sistemi intelligenti. La capacità di OmniBot di adattarsi a diversi domini e di interagire con un corpus di conoscenza dinamico lo rende uno strumento estremamente versatile, aprendo la strada a molteplici applicazioni future.

Nonostante i risultati ottenuti, alcune limitazioni sono emerse durante lo sviluppo e la sperimentazione di OmniBot. Ad esempio, la gestione della sicurezza e della privacy rimane una delle sfide più critiche. L'esecuzione di azioni concrete tramite un chatbot richiede un accesso sicuro a dati sensibili come le credenziali di pagamento o le informazioni personali. Implementare rigorose misure di protezione, come l'autenticazione a più fattori e la crittografia dei dati, sarà fondamentale per garantire che i potenziali rischi siano mitigati.

Per il futuro, lo sviluppo di OmniBot potrebbe concentrarsi su un'ulteriore espansione delle sue capacità di apprendimento, rendendolo capace di evolvere continuamente attraverso tecniche di apprendimento continuo. Un altro interessante sviluppo riguarda la possibilità di migliorare l'integrazione con i dispositivi IoT, rendendo OmniBot un vero e proprio hub per la gestione intelligente degli ambienti domestici e lavorativi.

In definitiva, il processo esaminato rappresenta un passo significativo verso la creazione di chatbot sempre più avanzati e integrati nel tessuto digitale del nostro mondo. Attraverso l'introduzione di architetture come la RAG e l'integrazione di action models, è possibile creare un sistema capace di assistere gli utenti non solo nella ricerca di informazioni, ma anche nell'esecuzione di azioni concrete. Tuttavia, il cammino verso l'integrazione di intelligenze artificiali sicure, etiche e altamente performanti è ancora lungo. I futuri sviluppi in questo campo dovranno concentrarsi non solo sull'ottimizzazione delle prestazioni, ma anche su una gestione responsabile delle tecnologie, affinché queste possano essere utilizzate in modo efficace e sicuro per il benessere della società.