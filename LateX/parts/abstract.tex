\chapter*{Abstract}
Lo sviluppo di tecnologie all'avanguardia ha trasformato profondamente le nostre vite. Negli ultimi anni, l'avanzamento esponenziale nelle tecniche di produzione di componenti hardware e nell'elaborazione di algoritmi di apprendimento automatico sempre più efficienti, ha favorito una rapida evoluzione nel campo del Machine Learning e del Deep Learning.
Di particolare interesse è il modo in cui queste tecnologie si interfacciano con il mondo del Natural Language Processing (NLP), e come questo approccio apra a possibilità quasi infinite. I Large Language Models (LLM), basati sull'architettura Transformer, hanno rivoluzionato il modo in cui le macchine comprendono e generano il linguaggio naturale, in più, l'integrazione con tecniche di Retrieval-Augmented Generation (RAG) ha ampliato ulteriormente le capacità e conoscenze di questi modelli, consentendo di combinare la generazione di linguaggio con il recupero di informazioni rilevanti da dataset estesi, rendendoli più applicabili a una vasta gamma di compiti.
Questa tesi non solo analizza le principali architetture alla base di questi modelli ma ne mostra molte delle loro applicazioni pratiche, mettendo in luce sia i punti di forza sia i loro lati oscuri. Inoltre, presenta una sperimentazione pratica attraverso lo sviluppo di un chatbot, "OmniBot", che implementa l'architettura RAG all'interno di una catena di pensiero personalizzata, denominata “Chain-of-Thoughts” (CoT). Verranno analizzati tutti i passi che hanno portato alla realizzazione di questo prototipo e verrà mostrato come questo approccio, oltre a migliorare la coerenza e la pertinenza delle risposte generate, può anche stabilire limitazioni mirate al comportamento del modello quando necessario, dimostrando, dunque, l'efficacia delle soluzioni discusse.
Infine, verranno presentati i risultati ottenuti attraverso test empirici e verranno discusse le possibili evoluzioni future di questo progetto.