\chapter{OmniBot: Plus Ultra}
Il progetto Plus Ultra rappresenta un'iniziativa di potenziamento dell'infrastruttura attuale di OmniBot, con l'obiettivo di eliminare i suoi punti deboli e massimizzare le sue caratteristiche di forza, spingendolo oltre i suoi limiti attuali. Questo programma prevede l'ottimizzazione di ogni componente del sistema, dalla gestione delle catene al miglioramento delle prestazioni del retriever, oltre all'integrazione di nuove tecnologie per incrementare la reattività e l'efficacia del modello.
Tutti i progressi realizzati nell'ambito di Plus Ultra saranno pubblicati e resi disponibili sul mio profilo GitHub \cite{github}, una volta completati e testati, in modo da favorire la trasparenza e contribuire alla comunità di sviluppatori e ricercatori che lavorano su tecnologie simili. A seguire sono presentati i quattro progetti principali che compongono Plus Ultra, ciascuno dei quali è dedicato a un aspetto specifico del nuovo sistema OmniBot: Plus Ultra.

\section{OmniAgent}
Una possibilità di sviluppo futuro per OmniBot consiste nella conversione dell'attuale sistema di catene, che definisce la Chain-of-Thoughts, in un sistema a grafo di agenti autonomi \cite{Wang_2024,Besta_2024,yao2023treethoughtsdeliberateproblem,xi2023risepotentiallargelanguage,guo2024largelanguagemodelbased,shao2024assistingwritingwikipedialikearticles}. Questo approccio consentirebbe di ottenere maggiore flessibilità e reattività del modello, migliorandone la capacità di adattamento a situazioni impreviste. L'implementazione di agenti autonomi potrebbe anche agevolare l'integrazione di nuove funzionalità, semplificando l'aggiornamento del sistema e riducendo i tempi di sviluppo, rendendo il codice più facile da mantenere e scalare.
In questo contesto, gli agenti autonomi potrebbero sfruttare strumenti specializzati per completare compiti specifici. Un esempio di tali strumenti potrebbe essere Tavily \cite{tavily}, che permette agli agenti di gestire informazioni, cercare risorse o comunicare con altri agenti in modo efficace. Questo sistema multi-agente potenzierebbe la capacità di OmniBot di operare in ambienti complessi e dinamici, rendendo il sistema più adattabile e performante. Tale miglioramento può essere implementato mediante l'utilizzo delle librerie di LangGraph \cite{langgraph}.

\subsection{ReAct e Parallel Thinking}
Un altro approccio interessante per migliorare OmniBot è l'introduzione della tecnica ReAct \cite{yao2023reactsynergizingreasoningacting}, che unisce ragionamento e azione in modo coordinato, permettendo al sistema di alternare tra riflessione e azioni concrete. Questo, non solo consente di reagire in modo più efficace a situazioni inaspettate, ma offre anche una maggiore efficienza operativa in scenari complessi.
In aggiunta, l'idea di Parallel Thinking \cite{snell2024scalingllmtesttimecompute,10.1093/pnasnexus/pgae233} potrebbe fornire un ulteriore strato di ottimizzazione. Esso permetterebbe al sistema di eseguire simultaneamente più percorsi di ragionamento, esplorando diverse soluzioni in parallelo. In questo modo, gli agenti autonomi sarebbero in grado di affrontare più compiti o problemi contemporaneamente, migliorando la capacità decisionale e riducendo i tempi di risposta. L'integrazione di queste tecniche potrebbe elevare OmniBot a un nuovo livello di intelligenza artificiale adattiva, potenziandone la velocità di elaborazione e la capacità di affrontare problemi complessi in tempo reale.

\section{OmniModal}
Un'altra area di sviluppo per OmniBot riguarda l'integrazione di modalità di input e output aggiuntive, con l'obiettivo di potenziare l'interazione sia con l'utente che con l'ambiente circostante. L'aggiunta di modalità sensoriali, come la visione artificiale, l'udito o persino il tatto, potrebbe consentire a OmniBot di acquisire informazioni più ricche e contestuali. Ad esempio, la visione artificiale permetterebbe al modello di analizzare immagini e video, facilitando l'identificazione di oggetti o la comprensione di scenari visivi. L'integrazione dell'udito consentirebbe di rispondere a input vocali, migliorando la reattività alle interazioni naturali e vocali degli utenti.
Con queste funzionalità, OmniBot sarebbe in grado di comprendere meglio il contesto in cui opera, interpretare segnali complessi dall'ambiente, e fornire risposte o azioni più precise e rilevanti. Questo approccio multimodale estenderebbe l'interfaccia uomo-macchina di OmniBot, rendendolo uno strumento ancora più potente e versatile per l'interazione con gli utenti in vari scenari, come assistenti personali, supporto in ambienti industriali o nel gaming.

\section{OmniAction}
Si valuta di includere un'innovativa espansione delle capacità del chatbot OmniBot, introducendo l'integrazione di un Large Action Model (LAM). Mentre gli LLM si concentrano prevalentemente sulla generazione di testo e sulla conversazione, l'aggiunta di un LAM permette al sistema di eseguire azioni concrete nel mondo reale. OmniAction potenzia il chatbot rendendolo capace non solo di elaborare risposte, ma anche di acquistare oggetti sul web, prenotare biglietti per eventi, interagire con dispositivi IoT, inviare e-mail, messaggi o persino automatizzare task complesse a nome dell'utente.

\subsection{Rischi e Sfide}
Questo tipo di potenziamento, sebbene altamente vantaggioso, presenta anche potenziali rischi \cite{shevlane2023modelevaluationextremerisks}. Primo fra tutti è la sicurezza. L'abilitazione di azioni autonome come l'acquisto online o l'invio di e-mail richiede accesso a dati sensibili come le credenziali di pagamento o contatti personali. Una gestione inadeguata della sicurezza potrebbe rendere il sistema vulnerabile a cyber attacchi o manipolazioni. È essenziale implementare protocolli di sicurezza avanzati, come la crittografia end-to-end e l'autenticazione a due fattori, per evitare che OmniBot venga compromesso.
Un altro rischio è rappresentato dall'abuso di potere: se mal configurato, il chatbot potrebbe eseguire azioni non desiderate o inappropriate, come acquisti non autorizzati o invio di messaggi a destinatari sbagliati. Un rigoroso sistema di controllo e validazione da parte dell'utente dovrebbe essere implementato per garantire che le azioni siano sempre consapevolmente confermate.

\section{OmniBotStudio}
Per facilitare lo sviluppo e la gestione di varianti personalizzate di OmniBot, si propone la creazione di OmniBotStudio, un'interfaccia grafica intuitiva e user-friendly per la configurazione e la personalizzazione del modello. Questo strumento consentirebbe agli utenti di selezionare e adattare facilmente le funzionalità di OmniBot, come la gestione delle catene, l'integrazione di nuove modalità sensoriali o l'implementazione di azioni specifiche.
In questo modo potrà essere possibile creare versioni personalizzate di OmniBot per diversi settori, come assistenti virtuali per il settore medico, chatbot per il supporto clienti o agenti virtuali per l'educazione. OmniBotStudio permetterebbe agli utenti di sfruttare appieno il potenziale di OmniBot, adattandolo alle proprie esigenze e creando soluzioni AI su misura per le proprie attività.