\chapter*{Ringraziamenti}
\markboth{\MakeUppercase{Ringraziamenti}}{\MakeUppercase{Ringraziamenti}}

Eccoci qua, alla fine di ciò che è stato un nuovo inizio. Fa strano pensare di dover scrivere i ringraziamenti di una tesi, ma smetterò di pensare per una volta e li scriverò.
Inizierò nel modo più normale possibile:

Ringrazio i miei genitori, che per 22 anni hanno dovuto sopportare il peso di avere un essere perfetto come figlio.
In particolare, ringrazio mia madre per avermi cresciuto ed avermi sempre sostenuto in ogni scelta.
Mio padre, invece, per avermi insegnato ad essere una persona sincera e a dire sempre ciò che penso.
Ringrazio mia sorella per essermi sempre stata accanto e per essere stata la sorella minore che tutti vorrebbero avere.

Ringrazio i miei nonni, sia quelli che ci sono, che hanno sempre creduto in me e, anche a distanza, hanno sempre fatto sentire la loro vicinanza,
sia quelli che non ci sono più, che anche ora mi fanno sentire il loro affetto.

Ringrazio i miei zii, da zia Sara che mi ha visto nascere e crescere, a zia Nicoletta che, oltre ad essere "la zia più bella del mondo", mi ha sempre dato buoni consigli.

Ringrazio i miei amici, a partire da quelli di lunga data: Mario per essere stato il mio matematico di banco per 5 anni e per avermi fatto scoprire un Signore degli Anelli diverso,
Michele per essere l'eroe che Satania non merita ma di cui ha bisogno e Paratore che è stato un ottimo compagno di start-up e mi ha insegnato il simonese.

Un ringraziamento a parte va a Tano, detto "Jaeger", sempre impegnato a salvare gli anziani o a spaccare ambulanze il quale è sempre stato un amico sincero e presente in ogni momento, sia in quelli di gioia che in quelli di difficoltà, dandomi il coraggio di continuare questa avventura.

Un posto speciale hanno i membri del Condominio che, dopo un anno passato tra i rimasti, mi hanno accolto tra loro.
Il più saggio è sicuramente il Compagno Mariano che mi ha installato Vampire Survivors, mi ha sempre accolto alle sue feste pazze fino alle 5 del mattino, ma soprattutto, è sempre stata una persona su cui contare.
Il secondo più saggio è Fabio che nonostante abbia provato a distruggermi la schiena, è sempre stato un punto di riferimento per i miei pensieri più profondi.
C'è poi Salvo, la Cirmi, che è sempre stato un amico fidato e presente, anche in azienda, accompagnandomi per prenotare il pranzo e prendere la crema al caffè.
Ovviamente Daniele che ha sempre assecondato il mio taglio di capelli e mi ha offerto un posto sicuro per la macchina.
Devo ringraziare pefforzah Mattia, il pozzallese che mi ha sempre fatto interessare ai suoi discorsi filosofici e ha sopportato le mie battute.
Il Malefico Michele che mi ha fatto vedere le cose da una prospettiva diversa e mi ha insegnato a prendere le cose con più leggerezza.
Giorgio che mi ha insegnato che se pago le tasse, allora posso usare l'aula studio come e quando voglio.

Ci sono poi i frequentatori dell'Aula Gaming, da Matteo, il Guardiano del Tavolo la cui presenza in aula è sempre stata l'unica certezza del DMI
e Edoardo, il catenoto che in questi anni mi ha sempre strappato un sorriso con i suoi meme di qualità.

Ringrazio la mia bellissima Kia Rio del 2013 che mi ha accompagnato in tutti i viaggi e mi ha permesso di dire "I drive" con orgoglio.

Ringrazio il mio spettacolare Acer Aspire 7 che nonostante qualche render più folle o qualche rete neurale senza senso non è mai esploso.

Devo ringraziare il mio assistente di coding, il mio consulente personale, il mio compagno più paziente: ChatGPT, che mi ha aiutato nei momenti di difficoltà o di noia.

Un sentito ringraziamento va al Prof. Sebastiano Battiato, relatore di questa tesi, per avermi dato l'opportunità di lavorare su questo progetto, offrendo il suo prezioso supporto e la sua guida.
Un ringraziamento speciale va inoltre all'Ing. Mario Barbera, correlatore e tutor durante il mio percorso presso l'azienda Intellisync, la quale ha reso possibile questa esperienza.
Desidero infine esprimere la mia gratitudine a Vanessa Castorina, per la sua costante disponibilità e per avermi offerto l'opportunità di iniziare lo stage presso l'azienda, evento determinante per la realizzazione di questo lavoro.

\bigskip

Sono stati anni particolari, pieni di alti e bassi, con giornate a volte al limite della realtà, ma alla fine grazie a tutti voi sono riuscito a raggiungere questo traguardo senza esaurire completamente.

%Probabilmente ho scritto i ringraziamenti più lunghi e allo stesso tempo più inutili della storia, ma ho quasi finito.
%Non sono mai stato obbligato a scriverli, ma per qualche motivo ho sempre pensato a cosa avrei voluto scrivere fin da quando ho iniziato l'università, forse per motivarmi un po' e immaginare di arrivare a questo punto.
%Nel corso del tempo questi hanno assunto forme diverse, a seconda del periodo sono state tante le persone che avrei voluto ringraziare o che avrei voluto cancellare dall'esistenza.
%Per questo motivo, ma soprattutto poiché sono una persona che quando promette o dice di fare qualcosa, prima o poi la fa, ho deciso di chiudere con un ultimo ringraziamento.

%Ringrazio la mia socia, non per essermi sempre stata accanto in tutto questo tempo, perché non è vero, ma per avermi insegnato la lezione più importante della mia vita, vale a dire che non ci si deve mai fidare di nessuno.